% !Mode:: "TeX:UTF-8"
\chapter{本项目的数学模型}

\section{车辆路径运输问题模型}
 Dantzig 和 Ramser 于1959年首次提出了车辆路径规划问题(VRP),这个问题是指不定量的顾客,
每个顾客有不同的需求,一个或多个物流中心向客户提供所需商品,并且由一个或多个配送员分配顾客的货物,并寻找合适的
配送路径,在满足客户需求的大前提下,能够完成例如路径最短,时间花费最少等成本消耗最小的目的。
\subsection{模型一:带有容量约束的车辆路径问题(CVRP)}
记顾客集合 $M = \{1,2, \cdots,n\}$
\par 
记顶点集合 $V = M \cup \{ 0,n+1\}$
\par
记车辆数量 D
\par
记车辆总容量 K
\par
记顾客j的需求 $R_j$
\par
记路径成本 $x_{ij}$
\par
目标函数如下: 
\newline
\begin{align}
    min\quad f = \sum_{i=0}^{n+1} \sum_{j=0}^{n+1} x_{ij}\omega_{ij} \label{mbhs_1}
\end{align}
\begin{numcases} {st.}
    \sum_{\substack{j=1 \\ j \ne i}}^{n+1} \omega_{ij}=1 \quad \forall i \in M \label{cys_1}\\
    \sum_{\substack{i=0 \\ k \ne i}}^{n+1} \omega_{ik} = \sum_{\substack{j=1 \\ k \ne i }}^n+1 \omega_{kj} \quad \forall k \in M \label{llys_1}\\
    \sum_{j=1}^n \omega_{0j} \Subset D \label{chys}\\ %出弧约束
    y_i + \omega_{ij} q_j - k(1-\omega_{ij}) \le y_i  \quad \forall i,j \in V \label{xczhl_1}\\
    q_j \le y_i \le Q \quad \forall i \in N \label{rlys}\\ %容量约束
    x_{ij} \in \{ 0,1\} \quad \forall i,j \in N \label{lyys_1} %0,1约束
\end{numcases}
\newline
上式中(\ref{mbhs_1})是目标函数,代表合法路径的成本最小化,(\ref{cys_1})代表每个点除起点外都要出一次,(\ref{llys_1})代表流量约束,指每个点出一次必须进一次,
(\ref{chys})代表起点出弧数小于车辆数量,而(\ref{xczhl_1})代表合法路径不存在回路,(\ref{rlys})表示需求不超过车的容量。(\ref{lyys_1})表示$\omega_{ij}$满足0,1约束。
\par
对于本文例子的应用,经过调查走访发现大学城南门各饭店小吃街的各个商户的外卖商品样式质量相似,所以各个顾客的需求可近似为1,其店铺规模普遍偏小,每家商户通常配备1名配送员,一名配送员通常服务1家门店
所以D = 1,K = 1,故模型可以进一步简化为模型二
\subsection{模型二:TSP问题模型}
记赋权图G=(V,E),V为顶点集 $V=\{1,2, \cdots,n \}$,E为边集,各顶点间的距离$d_{ij}$已知n为地点数量。
车辆路径运输模型如下:\\
\begin{equation}
    \begin{aligned}
        min\quad f=\sum_{i=1}^n \sum_{j=1}^n\omega_{ij} x_{ij} \label{mbhs} %目标函数
    \end{aligned}
\end{equation}
\begin{numcases} {st.}
    \sum_{\substack{i=1\\1 \le j\le n}}^n\omega _{ij} = 1 \quad \forall  i \in V \label{jys} \\ %进约束
    \sum_{\substack{j=1\\1 \le i\le n}}^n\omega _{ij} = 1 \quad \forall  j \in V \label{cys} \\ %出约束
    \sum_{i=1}^n \omega_{ij}=\sum_{k=1}^n \omega_{jk}\quad \forall  j \in V \label{llph} \\ %流量平衡
    \sum_{i\in S}\sum_{j\in \bar{S}}\omega _{ij} \ge 1 \quad \forall S \Subset V,S \ne \varnothing \label{xczhl}\\ %消除子回路
    \omega_{ij} \in \{0,1\} \quad \forall i,j \in V  \label{lyys} \\ %0,1约束 
    \omega _{ii} \ne 1 \quad \forall i \in V \label{tys} %图约束,代指自己不能通向自己
\end{numcases}
\newline
上式中$\omega_{ij}$代表最终路径中是否存在从i到j的通路,$x_{ij}$代表从i到j所需代价。
(\ref{mbhs})是目标函数,即通过所有节点后代价最小。目标函数下面为各个约束条件,其中(\ref{jys})说明任意节点的进度为1,
(\ref{cys})说明任意节点的出度为1,(\ref{jys})和(\ref{cys})规定合法路径中不存在分叉,(\ref{llph})是流量平衡约束,即合法路径流量
从一个节点到另外一个节点不会存在断流情况。(\ref{xczhl})是消除子回路约束,其中n个地点被任意分为S与$\bar{S}$,合法路径中应满足以
存在于S中的i点为起点,以存在于$\bar{S}$中的j点为终点的情况。(\ref{lyys})说明$\omega_{ij}$满足0,1约束。(\ref{tys})说明路径任意节点
并不存在一条自己到自己的通路。

\section{运输问题算法研究概况}
目前对于路径规划问题的算法分为四部分
\par
(1)精确式算法:
%   参考
%   用于求解VRP的精确式算法主要有: 分支定界法、整数线性规划法和动态规划法. 精确式算法主要适用问题结构简单的小规模VRP, 如谢                      %
%   涛等在求解背包问题时, 分支决策深度受限制, 求解效率不高, 最大求解规模为250. 对于大多数不具有明确良性结构的VRP, 精确式算法无法求得最优解,      %
%   且在实际计算中存在耗时巨大的情况. 为了提高求解效率和质量, 近年来, 学者们在应用其求解问题时, 一般会与其他算法进行混合. 张鹏乐等基于动态规划    %
%   的理论, 建立了快速动态规划算法, 能满足大规模CVRP的需求. 王晓琨等在求解绿色VRP时, 利用混合整数线性规划法建立模型, 运用元启发式算法进行求解    %

其精确式算法主要分为三部分即动态规划法,线性规划法,与分支定界法,但此三种方法的运行时间会随着问题的规模指数上升,对于大型问题普遍求解效率偏低,
经过近几年算法的发展,学者们对其进行了改进,王晓琨利用混合启发式快速算法与整数规划解决了电车充电需求问题,张鹏乐提出了一种求解大规模VCVRP问题的
快速动态规划模型即DPM-MST模型,并且改进后的算法求解质量受节点规模波动较小。
\par
(2)启发式算法:
%   参考
%   与精确式算法相比, 启发式算法发展于仿生学, 以寻求最优的可行解为目标, 能够解决大规模的VRP问题, 主要有节约法、改进节约法和插入检测               %
%   法. 李兵等采取节约法, 求解了规模数为20的DVRP问题. 李妍峰等在节约法的基础上进行了改进, 采取了两阶段的求解方法: 第1阶段采用节约法对             %
%   客户进行聚类; 第2阶段进行搜索求解.潘立军等在Solomon设计的前推值插入检测法的基础上, 提出时间插入检测, 有效求解了VRPMTW. 这一阶段的启发式算法.  %
%   及其改进都是以缩短计算时间提高效率和减少计算机占用内存为目标, 这对于问题目标的求解的作用比较微弱.                                           %

启发式算法相对于最优化算法而被定义,通常是一个基于直观或经验构造的算法,它可以在规定时间内给出一个问题的可行解,但求得的解与最优解的偏差无法被估计,
可行解的区间也无法被确切估计。其在VRP的应用主要由三类算法构成,即插入检测法,节约法,与改进节约法。启发式算法能够解决大规模VRP问题,尽管往往求出的
并非是最优解。此类算法的应用研究在十余年前较为流行,近几年研究文献逐步下降,例如孔媛设计了新的评价因子即最小评价因子并且结合了VSPA(接送顾客到机场的
车辆调度问题)问题的特点,基于顺序插入的方法构造路径。李兵采用了重新优化法方法即将车辆的出发点看作任务点使得问题可以看作可以使用静态问题算法求解的普
通车辆路径问题。但此阶段的启发式算法对于问题最优解的寻找作用较为薄弱。

\par
(3)元启发式算法:
%   参考
%   元启发式算法相较于启发式算法, 通过更加全面和彻底的搜索过程, 使得解的优良性有了较大的提高,学者在该领域的研究成果很丰富, 主要有禁忌搜索算法、     %
%   模拟退火算法、遗传算法、蚁群算法、粒子群搜索算法和混合算法。王茜在多车型多车槽VRP问题上使用了混合导引反应式禁忌                               %
%   搜索算法.李珍萍在多时间窗车辆路径问题上使用了智能水滴算法。                                                                               %

元启发式算法是启发式算法的改进,它是随机算法与局部搜索算法相结合的产物,相对于启发式算法,它求得的可行解质量有了较大的提升,它由遗传算法,蚁群算法,
模拟退火算法,粒子群算法,禁忌搜索算法与混合算法等六类算法组成。元启发式算法也是近几年研究成果最为丰富的一类算法。穆东等基于并行模型退火算法求解
时间依赖型车辆路径问题,罗勇等针对遗传算法的选择交叉变异步骤,提出了基于序的选择算子、基于最小代价树的交叉算子和基于随机点长度控制的变异算子。
其改进后的遗传算法收敛速度与全局搜索能力有了较大的提升。
\par
(4)神经网络:

除了以上常用的几种元启发式算法以外, 还有一些文献也提到了神经网络算法, 学者们将人工智能的方法应用到VRP问题中, 取得了较好的效果. 路径规划更
加合理, 速度也更快. 随着环境的动态性和信息的未知性, 通过神经网络学习的方式, 对未知环境进行训练式探索, 是一种较好的求解方法。而本文的研究方法
主要为神经网络,下面对该类方法进行较为详细的介绍。

