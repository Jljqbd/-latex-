% !Mode:: "TeX:UTF-8"

% 中英文摘要
\begin{cabstract}
    物流企业对我国物流运输乃至经济的发展有着了不可磨灭的贡献,有了物流企业其他实体行业的商品才会有生产和大量销售的可能,运输互动作为物流的一个组成部分,近年来收到了广泛的研究与讨论,其较为热门的解决方法,遗传算法,模拟退火算法,Hopfield网络也成为了各个领域研究的焦点。
    \par
    本论文首先介绍了运输问题的数学模型,然后介绍了用于解决运输问题的算法,如Hopfield神经网络,智能算法以及确定性算法,如动态规划,分支限界法。本论文应用的方法是是模拟退火算法的改进型,下面统称为GASA-Hopfield算法,首先将模拟退火算法于Hopfield神经网络相结合,并且为了改善模拟退火的收敛速度,运行速度,解的质量,本论文重写并改进了“选择”函数部分,并且将变异率由定值改为变量,将程序的退出条件改进为一定代数解不改变即退出程序,改进后的程序收敛速度,运行时间,解的平均质量都有了明显改善。
    \par
    为了将GASA-Hopfield应用于实例中,本论文首先采用爬虫技术动态的爬取指定经纬度地图数据,然后采用Dijkstra算法,交点求解算法生成了图的邻接矩阵。最后带入到程序中得到最终结果,并将结果在,最优解,平均解,求解时间,收敛代数四个维度与常见的解决TSP问题的算法进行对比,证明了算法的优越性。
\end{cabstract}

\begin{eabstract}
    Logistics companies have made an indelible contribution to the development of my country’s logistics and transportation and even the economy. Only with logistics companies’ products in other physical industries will it be possible to produce and sell in large quantities. Transportation interaction, as an integral part of logistics, has received extensive attention in recent years. Research and discussion, its more popular solutions, genetic algorithm, simulated annealing algorithm, and Hopfield network have also become the focus of research in various fields.
    \par
    This paper first introduces the mathematical model of transportation problems, and then introduces the algorithms used to solve transportation problems, such as Hopfield neural networks, intelligent algorithms, and deterministic algorithms, such as dynamic programming and branch and bound methods. The method used in this paper is an improved type of simulated annealing algorithm, which is collectively referred to as the GASA-Hopfield algorithm below. First, the simulated annealing algorithm is combined with the Hopfield neural network, and in order to improve the convergence speed, running speed, and solution quality of simulated annealing, This paper rewrites and improves the "selection" function part, and changes the mutation rate from a fixed value to a variable, and improves the exit condition of the program to a certain algebraic solution without changing the program. The improved program convergence speed, running time, The average quality of the solutions has improved significantly.
    \par
    In order to apply GASA-Hopfield to an example, this paper first uses crawler technology to dynamically crawl the specified latitude and longitude map data, and then uses the Dijkstra algorithm, and the intersection solving algorithm to generate the adjacency matrix of the graph. Finally, bring it into the program to get the final result, and compare the results in the four dimensions of optimal solution, average solution, solution time, and convergence algebra with common algorithms for solving TSP problems, which proves the superiority of the algorithm. 
\end{eabstract}