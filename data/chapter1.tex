% !Mode:: "TeX:UTF-8"
\chapter{绪论}

\section{引言}

运输问题,一版被描述为寻找某一种最优的运输方案使得被运输品从始发地运输到多个目的地,使得运输方利益最大化。
我国的物流运输现状,近年来有着规模扩大,运量增加的趋势. 截止到2021年,我国近五年的社会物流总额年均增长率超6.5\%,
物流业总收入超10万亿元.但现阶段我国物流还存在着效率低,兼容性差,建设滞后,机制障碍的问题。
所以在现阶段在面对大量的公路,铁路,海运等物流运输要求时,可以在时间和仓储方面消耗最少的资源无疑会更有优势,
这也对现有的物流路径规划算法有了更高的要求。
\section{课题的应用背景}
近几年来我国物流企业不断崛起,整体呈现出规模巨大,科技水平提升,物流发展格局不断优化的特点,物流企业
对我国物流运输乃至经济的发展有着了不可磨灭的贡献,有了物流企业其他实体行业的商品才会有生产和大量销售的可能
而商品生产出来的目的就是为了被消费者所消耗,而大学生作为聚集且消费能力强的群体,导致大学校园以及周边的外卖快递的
物流运输十分的错综复杂。但本文以天津西青大学城为例,研究物流运输的路径最优问题。面积近90平方公里,入驻师生约40万人。
入驻高校有天津工业大学、天津师范大学、天津理工大学、天津师范大学津沽学院、天津大学软件学院、天津公安警官职业学院、
天津农学院、天津城建大学、天津商业大学宝德学院、天津教育招生考试院等高等教育机构、公安部天津消防研究所等科研机构、
天津团泊体育中心等体育远动场所发展起来的郊区新城,近年来物流运输请求的逐年增加,如何让商品流转于各个地方最终到达消费者手中
且能够在消费者可接受的时间内送达成为了平台和快递员与外卖员常常思考的问题。但在后文的讨论中我们就会发现物流运输在大学校园的应用其本质就是解决TSP问题。
为了增加代码的复用性,本文所用的代码可以通过经纬度自动搜集任意地区的地图矢量数据,使用本文的方法对路径进行优化。
\section{本文的内容及结构安排}
本文一共分为四章。
\par
第一章 绪论。主要介绍了本文的研究背景与相关领域的介绍。
\par
第二章 相关技术及其理论基础。介绍了运输问题的理论模型以及解决常见运输问题的研究算法。
%\par
%第三章 神经网络原理及其算法。介绍了神经网络的基本概念,以及详细介绍了hopfield神经网络的模型,工作方式以及稳定性。
%\par
%第四章 解决路径规划问题的其他优化算法。介绍了除神经网络外其余已经成熟的解决路径规划的算法。
\par
第三章 数据爬取 介绍了本文例子如何从网站爬取并且对数据进行转化清洗最终转化为图的数据结构。
\par
第四章 西青大学城运用GASA-Hopfield 算法求解运输问题。实现本文的例子,并且将集成的 Hopfield 路径规划算法与其余算法在时间空间性能稳定性上进行比较。

